% ----------------------------------------------------------------------------------------------
\newpage
\chapter{Overview}
The QFCL Random C++ library covers routines used for generating samples from various probability distribution.


\section{\index{Engines}Engines}
Engines generate random numbers with certain properties. The C++ standard library, the BOOST libraries provide various random
number engines. The engines in QFCL adhere to the same interfaces and are thus interoperable with C++ and BOOST random related routines. 

The Engines provided in QFCL have two major benefits:
\begin{itemize}
\item They are thread safe.
\item These engines can generate parallel independent random streams.
\end{itemize}
Both features are requires for efficient parallel random number generation and in parallel Monte Carlo engines.

\subsection{Interface}
Engines provide the following interface
\begin{itemize}
\item operator() for generating random numbers.
\item max() that returns the higher random number that can be generated.
\item min() that returns the lowest random number that can be generated.
\end{itemize}

\section{\index{Distribution}Distribution}
todo


\section{Generating variates}
todo

\begin{lstlisting}[ language=C++,
                    caption=Creating Vanilla call price sampler,
                    emph={gbm_npv_vanilla_call},
                    emphstyle=\bf\color{blue}]
    
    typedef qfcl::random::cpp_rand ENG;
    typedef qfcl::random::gbm_npv_vanilla_call DIST;
    typedef qfcl::random::variate_generator< ENG, DIST > RNG;
    
    double S0 = 100;
    double vol = 0.25;
    double yield = 0.05;
    double r = 0.05; 
    double strike = 110;
    double t = 1.5;
    
    DIST call( S0, vol, yield, r, strike, t );
    ENG eng;
    RNG rng(eng,call);
    
    for (int i=0;i<1000; ++i) 
    	std::cout << rng() << "\n";
\end{lstlisting}
