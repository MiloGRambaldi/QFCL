% ------------------------------------------------------------------------------------------------
%
% ------------------------------------------------------------------------------------------------
\sublibfunction{Description}
The distribution uniform\_0ex\_1ex\_distribution and it's variants are used to generate samples
from uniform distributions. The ex and in keyword indicate exclusion of inclusion of 0 and/or 1 in the generated 
samples.

\begin{tabular}{ | l | l | }
  \hline
  distribution & range \\
  \hline                        
  uniform\_0ex\_1ex & $(0,1)$ \\
  uniform\_0in\_1ex & $[0,1)$ \\
  uniform\_0ex\_1in & $(0,1]$ \\
  uniform\_0in\_1in & $[0,1]$ \\
  \hline  
\end{tabular}

% ------------------------------------------------------------------------------------------------
%
% ------------------------------------------------------------------------------------------------
\subsection{uniform\_0ex\_1ex}
\sublibfunction{Synopsis}
\begin{lstlisting}[language=C++]
// Include files
#include <qfcl/random/distribution/uniform_0ex_1ex.hpp>

// Types
template<class RealType = double>  class uniform_0ex_1ex_distribution;

// convenience typedef
typedef uniform_0ex_1ex_distribution <double> uniform_0ex_1ex;
\end{lstlisting}


% ------------------------------------------------------------------------------------------------
%
% ------------------------------------------------------------------------------------------------
\subsection{uniform\_0ex\_1in}
\sublibfunction{Synopsis}
\begin{lstlisting}[language=C++]
// Include files
#include <qfcl/random/distribution/uniform_0ex_1in.hpp>

// Types
template<class RealType = double>  class uniform_0ex_1in_distribution;

// convenience typedef
typedef uniform_0ex_1in_distribution <double> uniform_0ex_1in;
\end{lstlisting}


% ------------------------------------------------------------------------------------------------
%
% ------------------------------------------------------------------------------------------------
\subsection{uniform\_0in\_1ex}
\sublibfunction{Synopsis}
\begin{lstlisting}[language=C++]
// Include files
#include <qfcl/random/distribution/uniform_0in_1ex.hpp>

// Types
template<class RealType = double>  class uniform_0in_1ex_distribution;

// convenience typedef
typedef uniform_0in_1ex_distribution <double> uniform_0in_1ex;
\end{lstlisting}


% ------------------------------------------------------------------------------------------------
%
% ------------------------------------------------------------------------------------------------
\subsection{uniform\_0in\_1in}
\sublibfunction{Synopsis}
\begin{lstlisting}[language=C++]
// Include files
#include <qfcl/random/distribution/uniform_0in_1in.hpp>

// Types
template<class RealType = double>  class uniform_0in_1in_distribution;

// convenience typedef
typedef uniform_0in_1in_distribution <double> uniform_0in_1in;
\end{lstlisting}











% ------------------------------------------------------------------------------------------------
%
% ------------------------------------------------------------------------------------------------
\sublibfunction{Example}
\begin{lstlisting}[language=C++,caption=Sample C++ code: function interface,numbers=left]
#include <qfcl/random/engine/mersene_twister.hpp>
#include <qfcl/random/distribution/uniform_0in_1ex.hpp>
#include <qfcl/random/variate_generator.hpp>

int main()
{
    typedef qfcl::random::cpp_rand ENG;
    typedef qfcl::random::uniform_0in_1ex DIST;
    typedef qfcl::random::variate_generator< ENG, DIST > RNG;
    
    DIST u;
    ENG eng;
    RNG rng(eng,call);
    
    for (int i=0;i<1000; ++i) 
    	std::cout << rng() << "\n";
    return 0;
}
\end{lstlisting}

