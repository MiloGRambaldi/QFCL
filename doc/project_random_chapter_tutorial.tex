% ----------------------------------------------------------------------------------------------
\newpage
\chapter{Tutorial}
A tutorial that introduces the fundamental concepts required to use Qfcl.Random, 
and shows how to use Qfcl.Random to develop simple programs. 


\section{Headers and Namespaces}
All the code in this library is inside namespace qfcl::random. You can to bring distribution names into scope perhaps with a using namespace qfcl::random; declaration, but it's recommended that you use declarations like using qfcl::random::normal\_inversion; 


In order to generate samples from  a distribution some\_special\_distribution you will need to include either the
header <qfcl/random/distributions/some\_special.hpp> or the "include everything" 
header: <qfcl/random/distributions.hpp>.

\begin{lstlisting}[language=C++,caption=Including a "single distribution" header]
#include <qfcl/random/distributions/some_special.hpp>
\end{lstlisting}


\section{Distribution names and aliases}
All distribution are defined as templates with the real type  as argument and have names ending with "\_distribution".
E.g. the normal\_inversion\_distribution, is an distribution sampler that uses the inversion method to generate normal variates.

\begin{lstlisting}[language=C++,caption=]
// include file
#include <qfcl/random/distributions/normal_inversion.hpp>

// declaring a distribution
qfcl::random::normal_inversion_distribution<float> my_dist;
\end{lstlisting}

The real type of distribution defaults to double, and all distributions have a typedef omitting the \_distribution in the name for the double 
real type. The following declarations are all identical

\begin{lstlisting}[language=C++,caption=]
// verbose version
qfcl::random::normal_inversion_distribution<double> my_dist;

// using the default template type (double)
qfcl::random::normal_inversion_distribution<> my_dist;

// using the convenience typdef
qfcl::random::normal_inversion my_dist;
\end{lstlisting}